\cvsection{Experience}
\begin{cventries}
  \cventry
    {Postdoctoral Researcher}
    {University of Balearic Islands |  LIGO Scientific Collaboration | LISA Consortium}
    {Palma, Spain}
    {October 2020 - March 2021}
    {
      \begin{cvitems}
        \item {Led the design, development and review of a new infrastructure within the LIGO Software Library (LALSuite) to obtain individual subdominant modes from waveform models of gravitational waves (GW) in the frequency domain.}
        \item {Contributed to the parameter estimation reanalysis of the public GWTC-1 and O3a catalogues of GW events with the waveform models developed during the Ph.D. Extensive use of HPC resources and generation of results web pages with the PEsummary package.}
        \item {Contributed to the analysis of the LIGO-Virgo Collaboration (LVC) and performed parameter estimation runs used for the study of the first detection of a black-hole neutron-star merger:  \url{https://doi.org/10.3847/2041-8213/ac082e}.}
        \item {Developed an implementation of our waveform models in LALSuite capable of running on GPUs by using the CUDA language.}
        \item {Contributed to the development and review of the first phenomenological time domain models to be used in the LVC data analysis infrastructure.}
      \end{cvitems}
    }\vspace{-10pt}
    
  \cventry
    {Ph.D. Student; Supervisors: Prof. Sascha Husa, Prof. Alicia Sintes Olives}
    {}
    {}
    {September 2016 - July 2020}
    {
      \begin{cvitems}
        \item {Development of two new frequency domain waveform models to be used in the data analysis infrastructure of the LVC detectors.}
        \item {The models describe the subdominant mode content of both aligned-spin and precessing binary black holes systems.}
        \item {Production of numerical relativity (NR) simulations with the BAM code.}
        \item {Calibration of the model to NR simulations by producing data-driven non-linear fits across parameter space using Mathematica.}
        \item {Pionereed the description of mode-mixing in phenomenological waveform models, their extension to the extreme-mass-ratio-inspiral regime and the inclusion of the Multibanding technique to significantly reduce the computational cost of the models. Both are key features for the expected longer signals that will be detected by the LISA mission.}
        %\item {Build the precessing model on top of the aligned-spin by using the standard twisting-up technique.}
        \item {Implementation and review of both models for production level in LALSuite. Extensive use of C debuggers, Python, Bash and distributed computing.}
        \item {Both models are now the preferred ones used by the LVC to analyze events from the O3b period, having particularly contributed to the analysis of the first detection of a black-hole neutron-star merger.}
        \item {The three papers related to the models (I led two of them) have achieved 68 citations and have provided me with an h-index of 3 in less than a year. With the rest of short author list papers I accumulate 252 citations and an h-index of 7.  }
      \end{cvitems} 
    }\vspace{-10pt}

    \cventry
    {Ph.D. Research Stay; Supervisor: Prof. Vivien Raymond}
    {University of Cardiff}
    {Cardiff, United Kingdom}
    {October 2019 - December 2019}
    {
      \begin{cvitems}
        \item {Work in gravitational waves data analysis within the LVC.}
        \item {Parameter estimation studies of exceptional events detected in the O3a period with the new waveform model developed in the thesis.}
        \item {Use of the LALInference library and the Bilby Python package to perform Bayesian Inference runs on computer clusters. New features were added to these packages to handle the extra freedom of subdominant modes waveform models.}
        \item {Results showed that the new aligned-spin model developed in the thesis produced better constrained results in a much shorter timescale than previous models.}
        \item {Use of the Reduced-Order-Quadrature (ROQ) algorithm and the PyROQ package to produce for the first time a prototype for the ROQ basis of a waveform model with subdominant modes.}
      \end{cvitems}
    }\vspace{-10pt}

\cventry
    {Lab Technician, Research Fellow; Supervisor: Prof. Asunción Fernández Camacho}
    {Institute of Materials Science of Seville | Spanish National Research Council}
    {Seville, Spain}
    {May 2016 - July 2016}
    {
      \begin{cvitems}
        \item {Study of porous silicon through the electron microscope.}
        \item {Manufacture porous silicon through the spark plasma sintering method. }
        \item {Prepare samples of material to be analyzed in the transmission and scanning electron microscopes. }
        \item {Use the transmission electron microscope to produce high quality pictures of the resulting structure of the material.}
        \item {Use specialized software to analyze the pictures, study the structure and composition of the material and seek its best properties.}
      \end{cvitems}
    }\vspace{-10pt}
    
\cventry
    {Research Fellow; Supervisor: Prof. Juan de Dios Zornoza}
    {Corpuscular Physics Institute (IFIC) | University of Valencia}
    {Valencia, Spain}
    {October 2015 - November 2015}
    {
      \begin{cvitems}
        \item {Search of dark matter with the neutrinos telescopes Antares and KM3NeT.}
        \item {Production of simulated skies of the background neutrino flux through  Montecarlo simulations run on the Lyon's computer cluster.}
        \item {Calculation of the dark matter halo profile with the program CLUMPY for different channels of formation and theoretical profiles.}
        \item {Use of data analysis and statistics techniques to determine the expected number of signals and their likelihood.}
        \item {Analysis and visualizations performed using the ROOT data analysis framework developed at CERN.}
        \item {Results showed that a new theoretical profile led to a consistently lower upper limit for the number of events required to achieve a real observation.}
        \end{cvitems}
    }\vspace{-10pt}

\cventry
    {Summer Research Fellow; Supervisor: Prof. Antonio Bueno Villar}
    {University of Granada}
    {Granada, Spain}
    {July 2014 - October 2014}
    {
      \begin{cvitems}
        \item {Research activity within the Ultra High Energy Particles group.}
        \item {Studies by computer simulations for the improvement in the measurement of cosmic rays detected by the Pierre Auger Observatory.}
        \item {Analysis and visualizations performed using the ROOT data analysis framework developed at CERN.}
        \item {Results showed the current analysis for the time of arrival was not robust enough and neglected some important factors.}
        \item {Contributed talk for the Pierre Auger collaboration in Malargüe, Argentina.}
      \end{cvitems}
    }\vspace{-10pt}

\end{cventries}
